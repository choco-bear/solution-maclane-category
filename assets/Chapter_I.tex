\setcounter{chapter}{0}
\chapter{Categories, Functors, and Natural Transformations}
\setcounter{section}{2}
\section{Functors} % Section 3.

\begin{exercise} % 1
  Show how each of the following constructions can be regarded as a functor:

  The field of quotients of an integral domain; the Lie algebra of a Lie group.
\end{exercise}

\begin{exercise} % 2
  Show that functors $\mathbf{1} \to C$, $\mathbf{2} \to C$, and $\mathbf{3} \to C$ correspond respectively to objects, arrows, and composable pairs of arrows in $C$.
\end{exercise}

\begin{exercise} % 3
  Interpret ``functor'' in the following special types of categories:
  (a) A functor between two preorders is a function $T$ which is monotonic (i.e., $p \leqq p'$ implies $T p \leqq T p'$).
  (b) A functor between two groups (one-object categories) is a morphism of groups.
  (c) If $G$ is a group, a functor $G \to \Set$ is a permutation representation of $G$, while $G \to \Matr{K}$ is a matrix representation of $G$.
\end{exercise}

\begin{exercise} % 4
  Prove that there is no functor $\Grp \to \Ab$ sending each group to its center (consider $S_2 \to S_3 \to S_2$, the symmetric groups).
\end{exercise}

\begin{exercise} % 5
  Find two different functors $T : \Grp \to \Grp$ with object function $T(G) = G$ the identity for every group $G$.
\end{exercise}

\section{Natural Transformations} % Section 4.

\begin{exercise} % 1
  Let $S$ be a fixed set, and $X^S$ the set of all functions $h : S \to X$. Show that $X \mapsto X^S$ is the object function of a functor $\Set \to \Set$, and the evaluation $e_X : X^S \times S \nat X$, defined by $e_X(h, s) = h(s)$, the value of the function $h$ at $s \in S$, is a natural transformation.
\end{exercise}
\begin{solution}
  For each function $f : X \to Y$, define the function $f^S : X^S \to Y^S$ by $f^S(h) = f \circ h$ for every $h \in X^S$.
  Then, it is straightforward to check that this defines a functor $(-)^S : \Set \to \Set$.

  Next, we show that the evaluation $e_X : X^S \times S \nat X$ is a natural transformation.
  For each function $f : X \to Y$, we need to show that the following diagram commutes:
  \begin{cd}
    X^S \times S \ar[r, "e_X"] \ar[d, swap, "f^S \times S"] \& X \ar[d, "f"] \\
    Y^S \times S \ar[r, "e_Y"] \& Y
  \end{cd}
  For each $(h, s) \in X^S \times S$, we have
  \[ f(e_X(h, s)) = f(h(s)) = f^S(h)(s) = e_Y(f^S(h), s) = e_Y((f^S \times S)(h, s)). \]
  Thus, the diagram commutes, and $e_X$ is a natural transformation, as desired.
\end{solution}

\begin{exercise} % 2
  If $H$ is a fixed group, show that $G \mapsto H \times G$ defines a functor $H \times - : \Grp \to \Grp$, and that each morphism $f : H \to K$ of groups defines a natural transformation $H \times - \nat K \times -$.
\end{exercise}
\begin{solution}
  For any group homomorphism $\varphi : G \to G'$, define $H \times \varphi : H \times G \to H \times G'$ by $(h, g) \mapsto (h, \varphi(g))$.
  Then it is straightforward to check that this defines a functor $H \times - : \Grp \to \Grp$.

  Next, for each group homomorphism $f : H \to K$, define the morphisms $\tilde{f}_G : H \times G \to K \times G$ as $\tilde{f}_G(h, g) = (f(h), g)$, for each group $G$.
  Hence, it suffices to show that for each group homomorphism $\varphi : G \to G'$, the following diagram commutes:
  \begin{cd}
    H \times G \ar[r, "\tilde{f}_G"] \ar[d, swap, "H \times \varphi"] \& K \times G \ar[d, "K \times \varphi"] \\
    H \times G' \ar[r, "\tilde{f}_{G'}"] \& K \times G'
  \end{cd}
  For each $(h, g) \in H \times G$, we have
  \[ (K \times \varphi)(\tilde{f}_G(h, g)) = (K \times \varphi)(f(h), g) = (f(h), \varphi(g)) = \tilde{f}_{G'}(h, \varphi(g)) = \tilde{f}_{G'}((H \times \varphi)(h, g)). \]
  Thus, the diagram commutes, and $\tilde{f} : H \times - \nat K \times -$ is a natural transformation, as desired.
\end{solution}

\begin{exercise} % 3
  If $B$ and $C$ are groups (regarded as categories with one object each) and $S,T : B \to C$ are functors (homomorphisms or groups), show that there is a natural transformation $S \nat T$ if and only if $S$ and $T$ are conjugate; i.e., if and only if there is an element $h \in C$ with $Tg = h(Sg)h^{-1}$ for every $g \in B$.
\end{exercise}
\begin{solution}
  Say $\eta : S \nat T$ is a natural transformation.
  Then, by the naturality condition, for the unique object $\bullet$ of $B$, and for each group element $g \in B$, we have the following commutative diagram:
  \begin{cd}
    S \bullet \ar[r, "\eta_\bullet"] \ar[d, swap, "S g"] \& T \bullet \ar[d, "T g"] \\
    S \bullet \ar[r, "\eta_\bullet"] \& T \bullet
  \end{cd}
  Thus, we have $\eta_\bullet \cdot S g = T g \cdot \eta_\bullet$ for each $g \in B$, which implies that $T g = \eta_\bullet \cdot S \, g \cdot (\eta_\bullet)^{-1}$.
  Hence, $S$ and $T$ are conjugate via the group element $\eta_\bullet \in C$.

  Conversely, assume that $S$ and $T$ are conjugate via an element $h \in C$; i.e., $T \, g = h \cdot S \, g \cdot h^{-1}$ for each group element $g \in B$.
  Define the morphism $\eta_\bullet : S \, \bullet \to T \, \bullet$ as $\eta_\bullet = h$, where $\bullet$ denotes the unique object of $B$.
  Then, for each group element $g \in B$, we have $T \, g \cdot h = h \cdot S \, g$, which implies the naturality condition for $\eta$.
  Therefore, $\eta : S \nat T$ is a natural transformation, as desired.
\end{solution}

\begin{exercise} % 4
  For functors $S,T : C \to P$ where $C$ is a category and $P$ a preorder, show that there is a natural transformation $S \nat T$ (which is then unique) if and only if $S c \leqq T c$ for every object $c \in C$.
\end{exercise}
\begin{solution}
  Say $\eta : S \nat T$ is a natural transformation.
  Then, for each object $c \in C$, $\eta_c : S c \to T c$ gives us $S c \leqq T c$ in $P$.

  Conversely, assume that $S c \leqq T c$ for each object $c \in C$.
  Then, for each object $c \in C$, there exists a unique arrow $\eta_c : S c \to T c$ in $P$.
  Hence, it suffices to show that for each arrow $f : c \to c'$ in $C$, the following diagram commutes:
  \begin{cd}
    S c \ar[r, "\eta_c"] \ar[d, swap, "S f"] \& T c \ar[d, "T f"] \\
    S c' \ar[r, "\eta_{c'}"] \& T c'
  \end{cd}
  Since $P$ is a preorder, there is at most one arrow between any two objects.
  Hence, the diagram commutes, as both paths represent the unique arrow from $S c$ to $T c'$ in the preorder $P$.
  Therefore, $\eta : S \nat T$ is a natural transformation, as desired.
\end{solution}

\begin{exercise} % 5
  Show that every natural transformation $\tau : S \nat T$ Tdefines a function (also called $\tau$) which sends each arrow $f : c \to c'$ of $C$ to an arrow $\tau f : S c \to T' c$ of $B$ in such a way that $T g \circ \tau f = \tau (g \, f) = \tau g \circ S f$ for all composable pair $\left\langle g, f \right\rangle$.
  Conversely, show that every such function $\tau$ comes from a unique natural transformation with $\tau_c = \tau(1_c)$.
  (This gives an ``arrows only'' description of a natural transformation.)
\end{exercise}

\begin{exercise} % 6
  Let $F$ be a field. Show that the category of all finite-dimensional vector spaces over $F$ (with morphisms all linear transformations) is equivalent to the category $\Matr{F}$ described in \S2.
\end{exercise}

\section{Monics, Epis, and Zeros} % Section 5.

\begin{exercise} % 1
  Find a category with an arrow which is both epi and monic, but not invertible (e.g., dense subset of a topological space).
\end{exercise}

\begin{exercise} % 2
  Prove that the composite of monies is monic, and likewise for epis.
\end{exercise}

\begin{exercise} % 3
  If a composite $g \circ f$ is monic, so is $f$. Is this true of $g$?
\end{exercise}

\begin{exercise} % 4
  Show that the inclusion $\mathbf{Z} \to \mathbf{Q}$ is epi in the category $\Rng$.
\end{exercise}

\begin{exercise} % 5
  In $\Grp$ prove that every epi is surjective (Hint.
    If $\varphi : G \to H$ has image $M$ not $H$, use the factor group $H/M$ if $M$ has index $2$.
    Otherwise, let $\operatorname{Perm} H$ be the group of all permutations of the set $H$, choose three different cosets $M$, $Mu$ and $Mv$ of $M$, define $\sigma \in \operatorname{Perm} H$ by $\sigma(xu) = xv$, $\sigma(xv) = xu$ for $x \in M$, and $\sigma$ otherwise the identity.
    Let $\psi : H \to \operatorname{Perm} H$ send each $h$ to left multiplication $\psi_h$ by $h$, while $\psi'_h = \sigma^{-1} \psi_h \sigma$.
    Then $\psi\varphi = \psi'\varphi$, but $\psi \neq \psi'$
  ).
\end{exercise}

\begin{exercise} % 6
  In $\Set$, show that all idempotents split.
\end{exercise}

\begin{exercise} % 7
  An arrow $f : a \to b$ in a category $C$ is \emph{regular} when there exists an arrow $g : b \to a$ such that $f \, g \, f = f$.
  Show that $f$ is regular if it has either a left or a right inverse, and prove that every arrow in $\Set$ with $a \neq \emptyset$ is regular.
\end{exercise}

\begin{exercise} % 8
  Consider the category with objects $\left\langle X, e, t \right\rangle$, where $X$ is a set, $e \in X$, and $t : X \to X$, and with arrows $f : \left\langle X, e, t \right\rangle \to \left\langle X', e', t' \right\rangle$ the functions $f$ on $X$ to $X'$ with $f \, e = e'$ and $f \, t = t' \, f$.
  Prove that this category has an initial object in which $X$ is the set of natural numbers, $e = 0$, and $t$ is the successor function.
\end{exercise}

\begin{exercise} % 9
  If the functor $T : C \to B$ is faithful and $T f$ is monic, prove $f$ monic.
\end{exercise}

\section{Foundations} % Section 6.

\begin{exercise} % 1
  Given a universe $U$ and a function $f : I \to b$ with domain $I \in U$ and with every value $f_i$ an element of $U$, for $i \in I$, prove that the usual cartesian product $\prod_i f_i$ is an element of $U$.
\end{exercise}

\begin{exercise} % 2
  (a) Given a universe $U$ and a function $f : I \to b$ with domain $I \in U$, show that the usual union $\bigcup_i f_i$ is a set of $U$.

  (b) Show that this one closure property of $U$ may replace condition (v) and the condition $x \in U$ implies $\bigcup x \in U$ in the definition of a universe.
\end{exercise}
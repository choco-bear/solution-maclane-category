\setcounter{chapter}{0}
\chapter{Categories, Functors, and Natural Transformations}
\setcounter{section}{2}
\section{Functors} % Section 3.

\begin{exercise} % 1
  Show how each of the following constructions can be regarded as a functor:

  The field of quotients of an integral domain; the Lie algebra of a Lie group.
\end{exercise}
\begin{solution}
  For any integral domain $R$, define $F(R)$ as the field of fractions of $R$.
  For any ring homomorphism $f: R \to S$ between integral domains, define $F(f): F(R) \to F(S)$ as $\frac{a}{b} \mapsto \frac{f(a)}{f(b)}$ for $a, b \in R$ with $b \neq 0$.
  This construction preserves composition and identities, thus defining a functor from the category of integral domains to the category of fields.

  For any Lie group $G$, define $\mathrm{Lie}(G)$ as the Lie algebra of $G$.
  For any Lie group homomorphism $\phi : G \to H$, define $\mathrm{Lie}(\phi) : \mathrm{Lie}(G) \to \mathrm{Lie}(H)$ as the differential of $\phi$ at the identity element.
  This construction also preserves composition and identities, thus defining a functor from the category of Lie groups to the category of Lie algebras.
\end{solution}

\begin{exercise} % 2
  Show that functors $\mathbf{1} \to C$, $\mathbf{2} \to C$, and $\mathbf{3} \to C$ correspond respectively to objects, arrows, and composable pairs of morphisms in $C$.
\end{exercise}
\begin{solution}
  \begin{enumerate}[label=(\arabic*)]
    \item The category $\mathbf{1}$ has a single object $0$ and only the identity $1_0$.
      The following give us a bijective correspondence between the functors $\mathbf{1} \to C$ and the objects in $C$:
      \begin{itemize}
        \item Each functor $F : \mathbf{1} \to C$ defines an object $F(0)$ in $C$.
        \item Each object $c$ in $C$ defines a functor $F : \mathbf{1} \to C$ by setting $F(0) = c$.
      \end{itemize}

    \item The category $\mathbf{2}$ has two objects $0$ and $1$, and three arrows: the identities $1_0$ and $1_1$, and the single non-identity arrow $\downarrow : 0 \to 1$.
      The following give us a bijective correspondence between the functors $\mathbf{2} \to C$ and the arrows in $C$:
      \begin{itemize}
        \item Each functor $F : \mathbf{2} \to C$ defines an arrow $F(\downarrow) : F(0) \to F(1)$ in $C$.
        \item Each arrow $f : c \to c'$ in $C$ defines a functor $F : \mathbf{2} \to C$ by setting $F(0) = c$, $F(1) = c'$, and $F(\downarrow) = f$.
      \end{itemize}

    \item The category $\mathbf{3}$ has three objects $0$, $1$, and $2$, and six arrows: the identities $1_0$, $1_1$ and $1_2$, and the three non-identity arrows $\downarrow_1 : 0 \to 1$, $\downarrow_2 : 1 \to 2$, and their composite $\downarrow_2 \circ \downarrow_1 : 0 \to 2$.
      The following give us a bijective correspondence between the functors $\mathbf{3} \to C$ and the composable pairs of arrows in $C$:
      \begin{itemize}
        \item Each functor $F : \mathbf{3} \to C$ defines a composable pair of arrows $F(\downarrow_1) : F(0) \to F(1)$ and $F(\downarrow_2) : F(1) \to F(2)$ in $C$.
        \item Each composable pair of arrows $f : c \to c'$ and $g : c' \to c''$ in $C$ defines a functor $F : \mathbf{3} \to C$ by setting $F(0) = c$, $F(1) = c'$, $F(2) = c''$, $F(\downarrow_1) = f$, and $F(\downarrow_2) = g$.
      \end{itemize}
  \end{enumerate}
\end{solution}

\begin{exercise} % 3
  Interpret ``functor'' in the following special types of categories:
  (a) A functor between two preorders is a function $T$ which is monotonic (i.e., $p \leqq p'$ implies $T p \leqq T p'$).
  (b) A functor between two groups (one-object categories) is a morphism of groups.
  (c) If $G$ is a group, a functor $G \to \Set$ is a permutation representation of $G$, while $G \to \Matr{K}$ is a matrix representation of $G$.
\end{exercise}

\begin{exercise} % 4
  Prove that there is no functor $\Grp \to \Ab$ sending each group to its center (consider $S_2 \to S_3 \to S_2$, the symmetric groups).
\end{exercise}
\begin{solution}
  Suppose not, i.e., there exists a functor $Z : \Grp \to \Ab$ sending each group to its center.
  Define $f : S_2 \to S_3$ as $f (1 \, 2) = (1 \, 2)$, and define $g : S_3 \to S_2$ as $g(\sigma) = (1 \, 2)^{\operatorname{sgn} (\sigma)}$ for each $\sigma \in S_3$, where $\operatorname{sgn} (\sigma)$ is the sign of the permutation $\sigma$.
  Then, we have $g \circ f = 1_{S_2}$, by a simple calculation.

  Applying the functor $Z$, we have $Z(g) \circ Z(f) = Z(1_{S_2}) = 1_{Z(S_2)}$.
  However, since the center of $S_3$ is trivial, $Z(f)$ maps everything to the identity element of $Z(S_3)$, making $Z(g) \circ Z(f)$ also the trivial map.
  This contradicts the fact that $Z(g) \circ Z(f) = 1_{Z(S_2)}$.
  Therefore, such a functor cannot exist.
\end{solution}

\begin{exercise} % 5
  Find two different functors $T : \Grp \to \Grp$ with object function $T(G) = G$ the identity for every group $G$.
\end{exercise}
\begin{solution}
  Say $\varphi : C_3 \to C_3$ is defined as $[n] \mapsto [2n]$.
  $\varphi$ is then an involutive automorphism of $C_3$.
  Define $T(f) = [\varphi \circ] f [\circ \varphi]$, where $[\varphi \circ]$ and $[\circ \varphi]$ denote composition only when they are composable.
  Then $T$ is a functor $\Grp \to \Grp$ with object function the identity.
  However, $T$ is different from the identity functor since $T(f) \neq f$ where $f : C_3 \to C_6$ is defined as $[n] \mapsto [2n]$.
  Therefore, we have found two different functors with the same object function.
\end{solution}

\section{Natural Transformations} % Section 4.

\begin{exercise} % 1
  Let $S$ be a fixed set, and $X^S$ the set of all functions $h : S \to X$. Show that $X \mapsto X^S$ is the object function of a functor $\Set \to \Set$, and the evaluation $e_X : X^S \times S \nat X$, defined by $e_X(h, s) = h(s)$, the value of the function $h$ at $s \in S$, is a natural transformation.
\end{exercise}

\begin{exercise} % 2
  If $H$ is a fixed group, show that $G \mapsto H \times G$ defines a functor $H \times - : \Grp \to \Grp$, and that each morphism $f : H \to K$ of groups defines a natural transformation $H \times - \nat K \times -$.
\end{exercise}

\begin{exercise} % 3
  If $B$ and $C$ are groups (regarded as categories with one object each) and $S,T : B \to C$ are functors (homomorphisms or groups), show that there is a natural transformation $S \nat T$ if and only if $S$ and $T$ are conjugate; i.e., if and only if there is an element $h \in C$ with $Tg = h(Sg)h^{-1}$ for every $g \in B$.
\end{exercise}

\begin{exercise} % 4
  For functors $S,T : C \to P$ where $C$ is a category and $P$ a preorder, show that there is a natural transformation $S \nat T$ (which is then unique) if and only if $S c \leqq T c$ for every object $c \in C$.
\end{exercise}

\begin{exercise} % 5
  Show that every natural transformation $\tau : S \nat T$ Tdefines a function (also called $\tau$) which sends each arrow $f : c \to c'$ of $C$ to an arrow $\tau f : S c \to T' c$ of $B$ in such a way that $T g \circ \tau f = \tau (g \, f) = \tau g \circ S f$ for all composable pair $\left\langle g, f \right\rangle$.
  Conversely, show that every such function $\tau$ comes from a unique natural transformation with $\tau_c = \tau(1_c)$.
  (This gives an ``arrows only'' description of a natural transformation.)
\end{exercise}

\begin{exercise} % 6
  Let $F$ be a field. Show that the category of all finite-dimensional vector spaces over $F$ (with morphisms all linear transformations) is equivalent to the category $\Matr{F}$ described in \S2.
\end{exercise}

\section{Monics, Epis, and Zeros} % Section 5.

\begin{exercise} % 1
  Find a category with an arrow which is both epi and monic, but not invertible (e.g., dense subset of a topological space).
\end{exercise}

\begin{exercise} % 2
  Prove that the composite of monies is monic, and likewise for epis.
\end{exercise}

\begin{exercise} % 3
  If a composite $g \circ f$ is monic, so is $f$. Is this true of $g$?
\end{exercise}

\begin{exercise} % 4
  Show that the inclusion $\mathbf{Z} \to \mathbf{Q}$ is epi in the category $\Rng$.
\end{exercise}

\begin{exercise} % 5
  In $\Grp$ prove that every epi is surjective (Hint.
    If $\varphi : G \to H$ has image $M$ not $H$, use the factor group $H/M$ if $M$ has index $2$.
    Otherwise, let $\operatorname{Perm} H$ be the group of all permutations of the set $H$, choose three different cosets $M$, $Mu$ and $Mv$ of $M$, define $\sigma \in \operatorname{Perm} H$ by $\sigma(xu) = xv$, $\sigma(xv) = xu$ for $x \in M$, and $\sigma$ otherwise the identity.
    Let $\psi : H \to \operatorname{Perm} H$ send each $h$ to left multiplication $\psi_h$ by $h$, while $\psi'_h = \sigma^{-1} \psi_h \sigma$.
    Then $\psi\varphi = \psi'\varphi$, but $\psi \neq \psi'$
  ).
\end{exercise}

\begin{exercise} % 6
  In $\Set$, show that all idempotents split.
\end{exercise}

\begin{exercise} % 7
  An arrow $f : a \to b$ in a category $C$ is \emph{regular} when there exists an arrow $g : b \to a$ such that $f \, g \, f = f$.
  Show that $f$ is regular if it has either a left or a right inverse, and prove that every arrow in $\Set$ with $a \neq \emptyset$ is regular.
\end{exercise}

\begin{exercise} % 8
  Consider the category with objects $\left\langle X, e, t \right\rangle$, where $X$ is a set, $e \in X$, and $t : X \to X$, and with arrows $f : \left\langle X, e, t \right\rangle \to \left\langle X', e', t' \right\rangle$ the functions $f$ on $X$ to $X'$ with $f \, e = e'$ and $f \, t = t' \, f$.
  Prove that this category has an initial object in which $X$ is the set of natural numbers, $e = 0$, and $t$ is the successor function.
\end{exercise}

\begin{exercise} % 9
  If the functor $T : C \to B$ is faithful and $T f$ is monic, prove $f$ monic.
\end{exercise}

\section{Foundations} % Section 6.

\begin{exercise} % 1
  Given a universe $U$ and a function $f : I \to b$ with domain $I \in U$ and with every value $f_i$ an element of $U$, for $i \in I$, prove that the usual cartesian product $\prod_i f_i$ is an element of $U$.
\end{exercise}

\begin{exercise} % 2
  (a) Given a universe $U$ and a function $f : I \to b$ with domain $I \in U$, show that the usual union $\bigcup_i f_i$ is a set of $U$.

  (b) Show that this one closure property of $U$ may replace condition (v) and the condition $x \in U$ implies $\bigcup x \in U$ in the definition of a universe.
\end{exercise}
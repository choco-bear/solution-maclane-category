\newpage
\begin{center}
\fbox{
\parbox{13cm}{
  Copyright \textcopyright\the\year\ doyoung @ \href{https://github.com/choco-bear/}{https://github.com/choco-bear/}. \\
  Permission is granted to copy, distribute and/or modify this document under the terms of the GNU Free Documentation License, Version 3.0 or any later version published by the Free Software Foundation; with the Invariant Sections being just ``GNU Manifesto'' and just ``Prologue'', with no Front-Cover Texts, and with no Back-Cover Texts. A copy of the license is included in the section entitled ``GNU Free Documentation License''.
}}

%\vspace{48pt}
%\emph{Special thanks to the following contributers:}
\end{center}

\newpage

\phantomsection
\addcontentsline{toc}{chapter}{Prologue}
\chapter*{Prologue}

\phantomsection
\addcontentsline{toc}{section}{Introduction}
\section*{Introduction}
\markboth{\MakeUppercase{Prologue}}{\MakeUppercase{Introduction}}

``Categories for the Working Mathematician.'' The title itself stands as a monument in the landscape of modern mathematics. Saunders Mac Lane's classic text is not merely a collection of theorems; it is a gateway to a new language--one that unifies the disparate branches of mathematics and reveals the ``universal properties'' that govern them.

As an undergraduate student, I found myself captivated by the promise of this perspective. I wanted to see the common threads weaving through algebra, topology, and beyond. However, the ascent into this level of abstraction is steep. The diagrams are elegant, yet the arguments can be dense, and the leap from definition to application is often perilous for a newcomer.

This solution manual is the record of my personal journey through that landscape.

I am not a master of category theory delivering wisdom from on high; I am a fellow traveler, an undergraduate student grappling with arrows, functors, and natural transformations. In these pages, I have documented my efforts to decode the definitions, to bridge the gaps in proofs, and to make sense of the structures that Mac Lane laid out.

My goal is to make this rigorous material more accessible to others who are drawn to the subject. Whether you are working through the diagrams with pencil and paper or seeking to deepen your understanding of mathematical structures, I hope you find a companion in these solutions. They are written with the intent to clarify, to explain the ``why'' alongside the ``how,'' and to share the moments of clarity that come after hours of struggle.

So, let's turn the page. Let's chase the diagrams and explore the profound unity of mathematics. Let's embrace the abstraction, not for its own sake, but for the clarity it brings to the work of the mathematician.

Welcome to the journey.

\newpage
\phantomsection
\addcontentsline{toc}{section}{Notations}
\section*{Notations}
\markboth{\MakeUppercase{Prologue}}{\MakeUppercase{Notations}}
Throughout this solution, we will use the following notations consistently:
\begin{itemize}
  \item We denote the set of natural numbers by $\mathbf N$ (including $0$)
  \item We denote the set of integers by $\mathbf Z$
  \item We denote the set of positive integers by $\mathbf{Z}_+$
  \item We denote the set of rational numbers by $\mathbf Q$
  \item We denote the set of real numbers by $\mathbf R$
  \item We denote the set of complex numbers by $\mathbf C$
  \item We denote the group of units of a ring $R$ by $R^\times$, or $R^*$
  \item We denote the power set of a set $X$ by $\mathscr{P}(X)$
  \item We denote the cyclic group of order $n$ by $C_n$
\end{itemize}
The notations not listed here are consistent with those in the textbook.

\newpage
\phantomsection
\addcontentsline{toc}{section}{Some important points}
\section*{Some important points}
\markboth{\MakeUppercase{Prologue}}{\MakeUppercase{Some important points}}
There are a few important points to note here:
\begin{itemize}
  \item The solution is \emph{only} hosted on my GitHub page
    \begin{center}
      \url{https://github.com/choco-bear/solution-maclane-category}.
    \end{center} 
    If you find this document outside this page, you might have an outdated version of the solution which might have errors, so please be aware.
  \item I will update the solution irregularly.
  \item I will try to reflect known errata for the textbook as much as possible.
  \item If you found an error in the solutions, typos, bad grammar or want to give an advise on LaTeX formatting, etc., don't hesitate to open an issue or a pull request on my repo.
  \item To get the latest PDF file immediately, you can use the following link:
    \begin{center}
      \url{https://github.com/choco-bear/solution-maclane-category/releases}.
    \end{center}
%     \item If you want to cite this solution, please use the following BibTeX entry:
%         \begin{verbatim}
% @misc{doyoung2026maclane,
%   author = {doyoung},
%   title = {Solutions to "Categories for the Working Mathematician"},
%   year = {2026},
%   howpublished = {\url{https://github.com/choco-bear/solution-maclane-category}},
% }
%         \end{verbatim}
\end{itemize}

Best,

\begin{flushright}
doyoung @ \url{https://github.com/choco-bear/} \\
Department of Computer Science \& Engineering, Seoul National University \\
Updated \specialdate\today \\
\end{flushright}